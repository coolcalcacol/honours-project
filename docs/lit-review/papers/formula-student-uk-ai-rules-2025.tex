\section{Analysis: Formula Student UK AI Rules}\label{sec:fs-uk-ai-rules-2025}

\subsection{Document Overview}\label{subsec:document-overview}
\begin{itemize}
    \item \textbf{Title:} Formula Student UK AI Rules 2025 (relevant sections pertaining to Dynamic Events and ADS requirements)
    \item \textbf{Source:} Formula Student / IMechE (assumed)
    \item \textbf{Year:} 2025 (effective year)
    \item \textbf{Type:} Competition Rulebook
\end{itemize}

\subsection{Core Focus/Summary}\label{subsec:core-focus/summary}
The Formula Student AI (FS-AI) 2025 rules outline dynamic events (Skidpad, Acceleration, Autocross, Trackdrive) requiring an Automated Driving System (ADS) to navigate cone-demarcated courses.
A core ADS requirement is the Dynamic Driving Task (DDT), encompassing Object and Event Detection and Response (OEDR), path planning, vehicle control, and mission tracking.
This analysis focuses on the implications for SLAM-based cone mapping and GA-based path planning.
Penalties for deviations underscore the need for precision.

\subsection{Key Rules and Implications for SLAM and Path Planning}\label{subsec:key-rules-and-implications-for-slam-and-path-planning}

\subsubsection{For SLAM Algorithm Design}
\begin{itemize}
    \item \textbf{Cone-based Mapping:} SLAM must use cones as primary landmarks, including reliable detection, classification, and precise localisation.
    \item \textbf{Real-time Performance:} Continuous map and pose updates are necessary for dynamic racing.
    \item \textbf{Robustness:} Must handle varying lighting/weather (Rule D3).
    \item \textbf{Accuracy and Consistency:} Critical for avoiding penalties (Rule D9) and reliable path planning, especially for multi-lap events.
    \item \textbf{Map Representation:} Output should be easily consumable by the GA path planner (e.g., cone coordinates).
    \item \textbf{Initialization and Loop Closure:} Handle unknown environments and maintain long-term accuracy.
\end{itemize}

\subsubsection{For Genetic Algorithm-based Path Planning}
\begin{itemize}
    \item \textbf{Input from SLAM:} Directly uses SLAM-generated cone map for drivable area/constraints.
    \item \textbf{Path Representation:} Suitable path representation for GA evolution (e.g., waypoints, splines).
    \item \textbf{Fitness Function Design:} Crucial for minimizing lap time, adhering to boundaries (avoid D9 penalties), ensuring path smoothness/feasibility, fulfilling event-specifics (lap counts), and safety.
    \item \textbf{Constraint Handling:} Effective management of track limits and vehicle dynamics.
    \item \textbf{Exploration vs Exploitation:} Balance in genetic operators for optimal racing lines.
    \item \textbf{Real-time Adaptation (Advanced):} Potential for path adaptation based on evolving SLAM data.
\end{itemize}

\subsubsection{Specific Dynamic Event Requirements}
\begin{itemize}
    \item \textbf{D4 Skidpad:} Precise circular/figure-eight paths; SLAM maps tight geometry; GA fitness check must ensure lap count (D9.1.11) and smooth trajectories.
    \item \textbf{D5 Acceleration:} Simple straight path; SLAM identifies start/end; GA ensures corridor adherence.
    \item \textbf{D6 Autocross/Sprint:} Complex single lap; rapid, accurate SLAM mapping; GA finds optimal line, penalizing errors.
    \item \textbf{D8 Trackdrive (10 laps):} Emphasises SLAM consistency and GA's repeated efficient path planning; potential for map refinement and path adaptation.
\end{itemize}

\subsection{Challenges and Considerations}\label{subsec:challenges-and-considerations}
\begin{itemize}
    \item \textbf{Sensor Data Interpretation:} Reliable cone detection/differentiation under varying conditions (D3 Weather Conditions) is foundational for SLAM .
    \item \textbf{Computational Resources:} SLAM and GAs must be optimised for on-board processing.
    \item \textbf{Integration of SLAM and Path Planning:} Tight, low-latency integration is vital.
    \item \textbf{State Estimation Uncertainty:} SLAM uncertainty estimates could inform robust GA decisions (e.g., safety margins).
    \item \textbf{Dynamic Obstacles (Not Explicitly Mentioned for Cones):} Unexpected obstacles would require extensions beyond basic cone navigation.
\end{itemize}

\subsection{Relevance to Honours Project: "Investigating the use of Genetic Algorithms to optimise a path planning algorithm within the Context of a Formula Student Team"}\label{subsec:relevance-to-honours-project:-"investigating-the-use-of-genetic-algorithms-to-optimise-a-path-planning-algorithm-within-the-context-of-a-formula-student-team"}

\subsubsection{Input Relevance}
\begin{itemize}
    \item \textbf{Problem Definition:} The rules directly define the operational domain, constraints (cone-based tracks, specific manoeuvres), and objectives (lap times, penalty avoidance) for the path planning algorithm.
    \item \textbf{Performance Metrics:} The rules establish the criteria (e.g., speed, accuracy, completion of laps) against which the GA-optimised path planner will be evaluated.
    \item \textbf{Sensor Data Context:} Understanding the environment (cones as landmarks) informs the type of data the SLAM system (input to the path planner) must provide.
\end{itemize}

\subsubsection{Process Relevance}
\begin{itemize}
    \item \textbf{Algorithm Design:} The GA's fitness function must be designed to directly address rule compliance (e.g., minimizing penalties for cone hits, maximizing lap completion).
    \item \textbf{Simulation Environment Design:} The simulation environment created for testing must accurately reflect the track layouts and rules described (e.g., cone placement, track dimensions for Skidpad, Autocross).
    \item \textbf{Testing Scenarios:} The dynamic events (Skidpad, Autocross, Trackdrive) provide specific scenarios for testing the robustness and performance of the integrated SLAM and GA path planning system.
\end{itemize}

\subsubsection{Output Relevance}
\begin{itemize}
    \item \textbf{Validation of Path Planner:} The success of the GA-optimised path planner will be measured by its ability to generate paths that allow the ADS to perform successfully in rule-compliant simulations of the dynamic events.
    \item \textbf{Demonstration of Feasibility:} Adherence to these rules demonstrates the practical applicability of the GA approach to the Formula Student AI challenge.
    \item \textbf{Project Deliverables:} The rules shape the requirements for the final path planning solution, ensuring it is tailored to the specific context of a Formula Student team.
\end{itemize}

\subsection{Further Notes/Actionable Insights}\label{subsec:further-notes/actionable-insights}
\begin{itemize}
    \item The FS-AI rules provide a structured environment for developing and testing autonomous racing algorithms, pushing innovation in real-time perception, mapping, and planning.
    \item Addressing these rules contributes to broader research in robust SLAM for high-speed navigation and optimal path planning in complex, constrained environments.
    \item The specific focus on cone-based navigation is relevant to other applications where simple, repeated landmarks define operational areas.
    \item Ensure the simulation environment accurately models penalty zones and scoring according to D9.
\end{itemize}